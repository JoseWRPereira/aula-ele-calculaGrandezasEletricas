\section{Associação de Resistores em Série}

Dado um ramo qualquer de um circuito resistivo, sabe-se que todos os elementos nele contidos estão associados em série, pois, aplicada uma diferença de potencial às suas extremidades, a corrente que percorre esse ramo é a mesma em todos os componentes, havendo apenas um caminho entre os nós conectados às extremidades.

Ao aplicar uma diferença de pontencial não nula entre os nós $A$ e $B$, haverá uma corrente elétrica entre esses pontos. O valor da intensidade dessa corrente depende, além da própria tensão, da resistência que o ramo apresenta, como diz a 1ª Lei de Ohm.

Todo elemento inserido em série no ramo, acrescenta resistência a ele.
Todo ramo com mais do que um resistor, pode ser substituído por um ramo equivalente, sem que haja mudança dos parâmetros elétricos do circuito, desde que não se altere a resistência equivalente do ramo, como mostrado na Figura \ref{fig:rSerie}.

\begin{figure}[!h]
	\centering
	\caption{Ramo entre nós $A$ e $B$ }
	\includegraphics[scale=1.0]{fig-rSerie}
	\label{fig:rSerie}
\end{figure}


A Figura \ref{fig:rSerie}(a) mostra o ramo com dois resistores, que formam a resistência total do ramo, que é ilustrado na Figura \ref{fig:rSerie}(b) com a resistência equivalente.
Pode-se representar a resistência equivalente em um ramo da seguinte forma:

\begin{eqnarray}
R_1 + R_2 = R_{eq}
\end{eqnarray}

Para uma quantidade de resistores no ramo maior do que dois, a resistência equivalente é calculada da mesma forma:

\begin{eqnarray}
R_1 + R_2 + ... + R_n = R_{eq}
\end{eqnarray}










\section{Associação de Resistores em Paralelo}

A intensidade da corrente que circula em um ramo, depende da tensão aplicada aos seus nós e da resistência total do ramo. Assim, para o circuito da Figura \ref{fig:paralelo} (a) a corrente total depende apenas do valor de $V_{CC}$ e $R_1$. Não havendo mudanças nesses valores, também não haverá mudança no valor da intensidade de corrente, denotada $I_T$. A intensidade da corrente que flui da fonte é a mesma que atravessa a carga, com uma certa resistência e condutância.

\begin{figure}[!h]
	\centering
	\caption{Circuito em paralelo}
	\includegraphics[scale=1.0]{fig-paralelo}
	\label{fig:paralelo}
\end{figure}

Ao inserir outro ramo ao circuito, como mostrado na Figura \ref{fig:paralelo}(b), a intensidade da corrente no ramo de $R_1$ não é alterado, pois a tensão entre os pontos $A$ e $B$ não muda, bem como seu valor ôhmico, porém, a intensidade total da corrente muda, pois há agora um novo caminho, um novo ramo para circulação da corrente.

A mudança do circuito (a) para o (b), produz um novo caminho para a circulação da corrente. Pode-se dizer então que com a inclusão do novo ramo, conectado aos mesmos pontos do ramo já existente, há uma maior facilidade para a corrente fluir, pois agora são dois caminhos, aumentando a condutância entre os nós.

Pode-se obter a condutância total do circuito paralelo somando as condutâncias dos ramos, da seguinte forma:

\begin{eqnarray}
  G_T & = & G_{R_1} + G_{R_2}
\label{eqn:GT}
\end{eqnarray}

A forma mais comum é trabalhar com a resistência e não a condutância, então,
substituindo (\ref{eqn:GinvR}) em (\ref{eqn:GT}) pode-se obter a resistência total:

\begin{eqnarray}
  \frac{1}{R_T}	& = & \frac{1}{R_1} + \frac{1}{R_2} \nonumber\\
	\nonumber\\
	R_T & = & \frac{1}{\frac{1}{R_1} + \frac{1}{R_2} }
	\label{eq:paralelo}
\end{eqnarray}


Da mesma forma pode-se aplicar para circuitos com varios ramos em paralelo, basta somar as condutâncias:

\begin{eqnarray}
	G_T & = & G_1 + G_2 + ... + G_n
\end{eqnarray}

Analogamente para a resistência total temos:

\begin{eqnarray}
	\label{eq:nrparalelo}
	R_T & = & \frac{1}{\frac{1}{R_1} + \frac{1}{R_2} + ... + \frac{1}{R_n} }
\end{eqnarray}


\subsection{Associação em Paralelo - Apenas 2 resistores}

O cálculo de resistores em paralelo pode ser realizado manipulando a Equação (\ref{eq:paralelo}) da seguinte forma:

\begin{eqnarray}
	G_{eq} & = & G_{R_1} + G_{R_2} \nonumber\\
	\nonumber\\
	\frac{1}{R_{eq}}	& = & \frac{1}{R_1} + \frac{1}{R_2} \nonumber\\
	\nonumber\\
	\frac{1}{R_{eq}}	& = & \frac{R_1 + R_2}{R_1.R_2} \nonumber\\
	\nonumber\\
	R_{eq}	& = & \frac{R_1.R_2}{R_1 + R_2} \nonumber
\end{eqnarray}

\textbf{Atenção!} Essa forma só é válida para a associação de dois resistôres.


\subsection{Associação resistores em paralelo - Todos iguais}

Outra forma particular para calcular resistência equivalente é quanto os resistores em paralelo são \textbf{todos} de mesmo valor, então o cálculo pode ser realizado manipulado a Equação (\ref{eq:nrparalelo}) da seguinte forma:

\textit{Dados os seguintes resistores em paralelo e nomeados sequencialmente: $R_1$ = $R_2$ = ... = $R_n$. }

\begin{eqnarray}
	G_{eq} & = & G_1 + G_2 + ... + G_n \nonumber\\
	\nonumber\\
	\frac{1}{R_{eq}} & = & \frac{1}{R_1} + \frac{1}{R_2} + ... + \frac{1}{R_n}\nonumber\\
	\nonumber\\
	\frac{1}{R_{eq}} & = & n.\frac{1}{R} \nonumber\\
	\nonumber\\
	\frac{1}{R_{eq}} & = & \frac{n}{R} \nonumber\\
	\nonumber\\
	\frac{R_{eq}}{1} & = & \frac{R}{n} \nonumber\\
	\nonumber\\
	R_{eq} & = & \frac{R}{n}
\end{eqnarray}

\textbf{Atenção!} Essa forma só é válida para a associação de resistôres iguais.
